\section{Evaluation von SeaMonster}
Die Anforderungen an Sicherheits Modellierungstools wurden bereits aufgezeigt. Im Folgenden wird die Arbeit mit dem ausgewählten Tool Seamonster bewertet.

Bei der Modellierung mit Seamonster kam es zu einigen Problemen.
\begin{itemize}
\item Teilweise fehlerhafte und veraltete Dokumentation,
\item beschnittener Funktionsumfang im Vergleich zu Vorgängerversionen (keine Vulnerability Cause Graphs, Security Activity Graphs und Security Models mehr),
\item Attack Trees bieten die Möglichkeit Werte mit den einzelnen Knoten zu verknüpfen. Diese sind jedoch unübersichtlich, schlecht zu erreichen und nicht für automatisierte Berechnungen verwendbar,
\item es ist nicht möglich neue eigene Elemente zu erstellen,
\item allgemein schlechte Menüführung und GUI
\item Diagramme können nicht untereinander verlinkt werden,
\item kollaboratives Arbeiten ist nicht möglich.
\end{itemize}

Nach der Arbeit mit Seamonster kamen wir zu dem Schluss, dass es bei der Sicherheitsmodellierung kein sehr nützliches oder notwendiges Tool ist.

Viele der dargestellten Problemen und Anmerkungen betreffen auch die anderen Modellierungstools. In einem realen Szenario wäre vermutlich das Fehlen von kollaborativem Arbeiten am schwerwiegendsten, da die Sicherheitsmodellierung ein kreativer Prozess ist, der im Team durchgeführt werden sollte. Ein simples Whiteboard wäre vermutlich vorzuziehen.