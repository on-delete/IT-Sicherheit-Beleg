\section{Einleitung}
Im Folgenden soll ein Überblick über den Inhalt dieses Belegs gegeben werden. Weiterhin wird der Ablauf des Projektes skizziert und die Aufteilung der Arbeit in Gruppen erläutert.

\subsection{Bedrohungs- und Risikoanalyse}
Jüngste Ereignisse wie der Sony-Skandal von 2011, bei dem es unbekannten Angreifern gelang, in das Online-Netzwerk PlayStation Network (PSN) einzudringen und Daten von rund 77 Millionen Nutzern zu stehlen, zeigen, dass auch große Unternehmen bei der Umsetzung eines umfassenden und angemessenen Sicherheitskonzeptes nicht alle möglichen Bedrohungen bedacht zu haben scheinen. Sony wurde daraufhin zu einer Strafzahlung von 300.000 Euro verurteilt. Somit entstand dem Unternehmen durch diesen Vorfall nicht nur ein wirtschaftlicher Schaden, auch der Imageverlust ist ein nicht zu ignorierender Faktor in dieser Angelegenheit, besonders durch die Tatsache, dass dieser Angriff hätte verhindert werden können. 
\\
\\
Gerade durch die immense Ausbreitung des Internets und der zunehmenden Menge an Daten, die in digitaler Form verschickt oder gespeichert werden, sind auch kleinere Unternehmen gezwungen, sich vor bestehenden Risiken zu informieren und gegebenenfalls Gegenmaßnahmen einzuleiten. Das Risikomanagement bildet dabei die Grundlage, um ein umfassendes Sicherheitskonzept, angepasst an die jeweilige Situation des betrachteten Objektes, zu erstellen und einzuführen. Die Bedrohungs- und Risikoanalyse ist ein wichtiger Baustein des Risikomanagements und dient hauptsächlich zur Identifizierung und Bewertung der möglichen Gefährdungen. Solche können zum Beispiel aktive Angriffe sein, bei denen der Angreifer Sicherheitslücken ausnutzt oder Passwörter per Brut-Force Attacken ausspäht. Aber auch Datenverlust durch höhere Gewalt wie Brand oder Hochwasser können die Integrität des bestehenden Systems gefährden. Menschliches Versagen oder ungeschulte Mitarbeiter sind ebenfalls wichtige Faktoren, die betrachtet und eingeschätzt werden müssen.

\subsection{Projektaufgabe}
Im Rahmen des Moduls "IT-Sicherheitsmanagement" im Wintersemester 2014/15 an der Hochschule Zittau/Görlitz Fachbereich Informatik wurde ein Projekt mit dem Inhalt einer solchen Bedrohungs- und Risikoanalyse durchgeführt. 

\subsection{Ablauf}

\subsection{Vorgehensweise}

\subsection{Aufteilung}
\begin{table}[H]
	\sffamily
	\caption{Aufgabenverteilung}
	\tabulinesep = 1mm %bringt die Reihen etwas weiter auseinander, angenehmer zu lesen
	\centering
		\begin{tabu} to 0.9\textwidth {| X[1.5] | X[3] |}
		\hline
		\textbf{Name} & \textbf{Aufgabe}\\
		\hline 
		Kucera, Adam & \\
		\hline
		Krause, Andre & Kapitel Einleitung\\
		\hline
		Leuschner, Jens & \\
		\hline
		Mack, Tobias & \\
		\hline
		Michel-Suarez, Maria Belen & \\
		\hline
		Müssig, Daniel & \\
		\hline
		Riedel, Robert & \\
		\hline
		Wollstein, Romano & \\
		\hline
		Zoeke, Robert & Misuse Cases \& Evaluation von Seamonster\\
		\hline
	\end{tabu}
\end{table}