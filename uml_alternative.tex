\section{UML als adäquate Lösung?}
In diesem Abschnitt soll beschrieben werden, in wie weit UML als Alternative zu Tools wie zum Beispiel SeaMonster verwendet werden kann. Dazu sollen die Vor- und Nachteile dieser Lösung aufgezeigt werden. Zum Schluss soll bewertet werden, ob UML zur Risikoanalyse eingesetzt werden sollte oder nicht.

\subsection{Vorteile}
UML ist eine Modellierungssprache, die seit den 1990er Jahren vor allem in der Software Entwicklung eingesetzt wird. Da es bereits seit ca. 25 Jahren und von beinahe jedem Entwickler eingesetzt wird, verfügt UML über viele Tools, die verwendet werden können um UML-Diagramme zu erstellen. Es gibt für jedes Betriebssystem mindestens ein Tool, wobei so gut wie alle sehr leicht zu bedienen sind. Die weite Verbreitung dieser Sprache sorgt vor allem auch dafür, dass sie bereits in der Ausbildung gelernt wird und daher auch von einer breiten Masse verstanden wird.\\
UML umfasst ca. 13 Diagrammarten, was dafür sorgt, dass UML eine große Palette an Modellen unterstützt. Es bietet zum Beispiel die Möglichkeit UseCase-Diagramme zu zeichnen, die den MisUseCase-Diagrammen der Risikoanalyse entsprechen.

\subsection{Nachteile}
Bei der Risikoanalyse werden häufig sehr viele Daten benötigt. Diese alle in ein Diagramm unterzubringen ohne das die Lesbarkeit darunter leidet ist kaum möglich. Gerade auch die Tatsache, das bei der Analyse viele Berechnungen durchgeführt werden müssen mindert die Nutzbarkeit von UML.\\
Außerdem bietet UML auch nur eine begrenzte Menge an Zeichenelementen, wodurch zum Beispiel das Zeichnen von Attack-Trees nicht sinnvoll möglich ist.

\subsection{Bewertung}
Die große Anzahl der Diagrammarten sowie die weite Verbreitung der Sprache sprechen für UML. Jedoch überwiegen die Nachteile, da aufgrund fehlender Zeichenelemente und der Unübersichtlichkeit die Verwendung von UML erschwert wird. Auch fehlt die Verknüpfung zu einer Tabelle, in der weitere Einzelheiten und Anmerkungen festgehalten werden können. Aufgrund der überwiegenden Nachteile, ist von der Verwendung von reinem UML für die Risikoanalyse abzuraten.