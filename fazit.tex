\section{Fazit}
Die Arbeit an diesem Projekt hat vor allem gezeigt, dass es mit unter ziemlich schwierig sein kann, konkrete Bedrohungen für ein Objekt zu finden und diese auch entsprechend zu bewerten. Außerdem ist die Einstufung von Risiken und dem entstehenden Schaden von Unternehmen zu Unternehmen unterschiedlich. Es hat sich gezeigt, dass die Risiken und Bedrohungen oft gleich sind, es aber eine eingehende Analyse benötigt, um den optimalen Sicherheitsprozess für das jeweilige System zu finden.
\\
\\
Bei der Auswahl der Tools zur Unterstützung der Analyse sind schnell Probleme aufgetreten, wie etwa das Fehlen eines umfangreichen Angebots solcher Modelingprogramme. Außerdem bieten Open-Source Programme oft nicht den gewünschten vollen Umfang an Funktionen, wodurch man gewisse Einschränkungen in Kauf nehmen muss. Zwar gibt es auch die Möglichkeit, kommerzielle Software einzusetzen, welche durch einen besseren Support seitens des Herstellers und eines größeren Funktionsumfangs hervorstechen. Es hätte aber in keinem Verhältnis zu den Kosten/Nutzen in diesem knapp sechswöchigen Projekt gestanden, daher viel eine solche Lösung heraus.
\\
\\
Die Betrachtung der UML als eine adäquate Alternative zu den Modeling-Tools hat außerdem ergeben, dass es trotz der Vielzahl an Beschreibungselementen und der weiten Verbreitung der Sprache alleinig nicht sinnvoll einsetzbar ist. Zum einen fehlen entsprechende Verknüpfungen zu den Schadensberechnungstabellen, und zum anderen sind die MisUse-Cases bzw. AttackTrees abhängig von vielen Faktoren, die in das Diagramm mit einfließen müssten. Somit ist die Übersichtlichkeit gerade bei größeren Analysen nicht mehr gegeben.