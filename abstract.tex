\section*{Abstract}
Jüngste Ereignisse, bei denen Millionen Daten von Nutzern aus Online-Systemen gestohlen wurden, zeigen, dass die Bedeutung von Risikoanalysen und die Einrichtung von Schutzmaßnahmen stetig zunehmen. Gerade durch komplexer werdende Systeme und Software sowie die Globalisierung sehen sich viele Unternehmen immer stärkeren Bedrohungen ausgesetzt, die ohne eine durchdachte Sicherheitsstrategie viel Schaden anrichten können. 
\\
\\
In dieser Arbeit wird eine solche Bedrohungs- und Risikoanalyse exemplarisch für zwei Fallbeispiele mit Hilfe eines geeigneten Tools durchgeführt. Weiterhin wird die UML als eine mögliche Alternative zur Darstellung der Analyse untersucht. Außerdem findet eine kritische Betrachtung des eingesetzten Tools zur Analyse statt, bei der die Schwächen der Modelingsoftware aufgezeigt werden.