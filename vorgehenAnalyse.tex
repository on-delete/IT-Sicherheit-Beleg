\section{Vorgehen bei der Risikoanalyse}

Abbildung~\ref{5analysestufen} zeigt schematisch das Vorgehen bei der Risikoanalyse. Es gilt als erstes die zu schützenden \textit{Werte} zu \textit{identifizieren}. Danach ermittelt man die \textit{Bedrohungen}, welche für die vorher ermittelten Werte bestehen. Anschließend müssen die \textit{Schwachstellen} identifiziert werden, durch welche die Bedrohung wirksam werden kann. Anschließend kann man die \textit{Risiken bewerten} mittels der Formel
\\\\
$ Risiko = Bedrohung * Schwachstelle * Wert $
\\\\
und entsprechende \textit{Prioritäten setzten}, was geschützt werden muss. Ist dies geschehen, kann man entsprechende \textit{Gegenmaßnahmen finden}.

\begin{figure}[h]
\includegraphics[scale=0.8]{images/5analysestufen.pdf}
\caption{Schematisches Vorgehen bei der Risikoanalyse}
\label{5analysestufen}
\end{figure}

\\

\subsection{Werte identifizieren}
Um die zu schützenden Werte zu identifizieren gibt es mehrere Möglichkeiten. Der einfachste Ansatz ist des, dem Fluss des Geldes zu folgen. Objekte mit einem festen Preis in der Anschaffung können exakt mit Zahlen erfasst werden. Sie sind relativ leicht für die Wiederbeschaffung zu beziffern. 