\section{Schadensberechnung}
Anhand der MisUse Cases und Attack-Trees aus den vorigen Kapiteln werden die Risiko-Analyse-Tabellen  aufgestellt. Dabei werden die folgenden Rahmenbedingungen in die Betrachtung eingezogen:
\begin{itemize}
\item ...keine 24 Stunden Videoüberwachung haben.
\item ...kein Sicherheitsdienst vor der Eingangstür haben.
\item ...Chipkartenschlösser vor den wichtigen Räumen haben.
\item ...Unterschiedliche manuelle Schlüssel an den Büros haben.
\item ...Personal und Mitarbeitern die Sicherheitsrechtlinien beigebracht haben.
\end{itemize}

\noindent
Risiko-Analyse Tabellen sind nach folgenden Punkten aufgebaut:
\begin{itemize}
\item Identifikation der Werte.
\item Identifikation der Bedrohungen des Wertes.
\item Identifikation der Schwachstelle der Bedrohung.
\item Mit welcher Wahrscheinlichkeit tritt eine Bedrohung ein? 
\item Welche Auswirkung hat die Bedrohung?
\item Abschätzung der Konsequenzen.
\item Abschätzung  der Kosten des materiellen Schadens.
\item Identifikation der Maßnahmen.
\item Die Skala der Auswertung liegt zwischen 1 (niedrigste Kennziffer) und 5 (höchste Kennziffer). 
\item Die Skala des Risikos liegt zwischen 1 (niedrigste Kennziffer) und 25 (höchste Kennziffer), es ist die Vermehrung der Wahrscheinlichkeit mal der Schaden.
\end{itemize}

\subsection{Ergebnisse}
Im Folgenden sind die Ergebnisse für AutoVit und Fakultät Informatik an der HSZG dargestellt. Die Daten der Risiko Analyse beider Einrichtungen sind in etwa identisch, weil beide ähnliche Werte und Bedrohungen haben.

\begin{table}[H]
	\sffamily
	\caption{Hardware vs. Diebstahl und Malware}
	\tabulinesep = 1mm %bringt die Reihen etwas weiter auseinander, angenehmer zu lesen
	\centering
		\begin{tabu} to 0.9\textwidth {| X[1]  X[1, p] | X[1] | X[1] | X[1] | X[1] | X[1] |}
		\hline
		\textbf{Wert} & \textbf{Bedrohung} & \textbf{Schwachstelle} & \textbf{Wahrscheinlichkeit} & \textbf{Schaden} & \textbf{Risiko} & \textbf{Maßnahme}\\
		\hline 
		Kucera, Adam &  & & & & &\\
		\hline
	\end{tabu}
\end{table}

Die Wahrscheinlichkeit, dass ein Einbruch durch freien Zugang auf das Firmengelände und  nicht vorhanden sein von 24 Stunden Videoüberwachung passiert, ist hoch. Der Schaden, der angerichtet werden kann ist verhältnismäßig gering. Daraus resultiert ein niedriges Gesamtrisiko.
Die Wahrscheinlichkeit, dass es keine oder schlechtes Überprüfung des Inventars gibt, ist niedrig, weil der Hausmeister es regelmäßig kontrolliert, aber die Schäden sind hoch, weil das Material sehr viel wert ist. Daraus resultiert ein niedriges Gesamtrisiko.
Die Wahrscheinlichkeit, dass ein Einbruch durch Generalschlüssel für alle Mitarbeiter und  ungesicherte Türen passiert, ist niedrig und mittelmäßig, weil die meisten Büros ihr eigenes Schloss haben, die Türen haben keine besondere Sicherheitsvorrichtung. Der Schaden, der angerichtet werden kann ist verhältnismäßig gering. Daraus resultiert ein niedriges Gesamtrisiko.
Die Maßnahme um die oben genannten Bedrohungen zu vermeiden sind, einen sicheren Eingang oder Zaun bauen, Videoüberwachung einstellen, das Inventar kontrollieren, Sicherheitsvorrichtungen an die Türen montieren und Schulung für die Mitarbeiter zu geben.

\begin{table}[H]
	\sffamily
	\caption{Hardware vs. Diebstahl und Malware}
	\tabulinesep = 1mm %bringt die Reihen etwas weiter auseinander, angenehmer zu lesen
	\centering
		\begin{tabu} to 0.9\textwidth {| X[1]  X[1] | X[1] | X[1] | X[1] | X[1] | X[1] |}
		\hline
		\textbf{Wert} & \textbf{Bedrohung} & \textbf{Schwachstelle} & \textbf{Wahrscheinlichkeit} & \textbf{Schaden} & \textbf{Risiko} & \textbf{Maßnahme}\\
		\hline 
		Kucera, Adam &  & & & & &\\
		\hline
	\end{tabu}
\end{table}

Mit einer unsicheren Firmenhardware und Firmendaten auf BYOD Geräten ist das Risiko am höchsten. Obwohl die Wahrscheinlichkeit, dass die oben genannten Schachstellen eintreten niedrig sind. Die Leitung der Firma/Hochschule hat die Mitarbeiter bezüglich Sicherheitsrichtlinien geschult.

\begin{table}[H]
	\sffamily
	\caption{Serverkonfiguration vs. Bedrohungen}
	\tabulinesep = 1mm %bringt die Reihen etwas weiter auseinander, angenehmer zu lesen
	\centering
		\begin{tabu} to 0.9\textwidth {| X[1] | X[1] | X[1] | X[1] | X[1] | X[1] | X[1] |}
		\hline
		\textbf{Wert} & \textbf{Bedrohung} & \textbf{Schwachstelle} & \textbf{Wahrscheinlichkeit} & \textbf{Schaden} & \textbf{Risiko} & \textbf{Maßnahme}\\
		\hline 
		Kucera, Adam &  & & & & &\\
		\hline
	\end{tabu}
\end{table}

Im Prinzip muss ein Server richtig konfiguriert werden, andernfalls ist die Wahrscheinlichkeit, dass die Bedrohungen in Tabelle 2 eintreten noch größer. 
Die höchsten Gesamtrisiken der Serverkonfiguration sind Angriffe eines unsicheren Diensts und Malware-Einschleusen durch fehlende oder schlechte Firewall-Konfiguration. 
Gezielte Angriffe auf das System durch die schlechte Schulung der Mitarbeiter, stellen auch ein sehr hohes Risiko der Server Konfiguration dar.  DDoS Attacken können durch fehlende AntiVirus-Programme ausgelöst werden.

\begin{table}[H]
	\sffamily
	\caption{Schulung vs. Bedrohungen}
	\tabulinesep = 1mm %bringt die Reihen etwas weiter auseinander, angenehmer zu lesen
	\centering
		\begin{tabu} to 0.9\textwidth {| X[1] | X[1] | X[1] | X[1] | X[1] | X[1] | X[1] |}
		\hline
		\textbf{Wert} & \textbf{Bedrohung} & \textbf{Schwachstelle} & \textbf{Wahrscheinlichkeit} & \textbf{Schaden} & \textbf{Risiko} & \textbf{Maßnahme}\\
		\hline 
		Kucera, Adam &  & & & & &\\
		\hline
	\end{tabu}
\end{table}

Eine professionelle Schulung der Mitarbeiter kann Risiken vermeiden.
Unsichere Passwörter von Mitarbeitern sind anfälliger für Hackers. Wirtschaftsethik spielt  eine wichtige Rolle, sonst können interne Sozial Engineering Informationen, durch Missbrauch der Daten zweckentfremdet werden.

\begin{table}[H]
	\sffamily
	\caption{Industriespionage vs. Bedrohungen}
	\tabulinesep = 1mm %bringt die Reihen etwas weiter auseinander, angenehmer zu lesen
	\centering
		\begin{tabu} to 0.9\textwidth {| X[1] | X[1] | X[1] | X[1] | X[1] | X[1] | X[1] |}
		\hline
		\textbf{Wert} & \textbf{Bedrohung} & \textbf{Schwachstelle} & \textbf{Wahrscheinlichkeit} & \textbf{Schaden} & \textbf{Risiko} & \textbf{Maßnahme}\\
		\hline 
		Kucera, Adam &  & & & & &\\
		\hline
	\end{tabu}
\end{table}

Heutzutage sind sowohl die Industriespionage als auch Diebstähle deutlich gestiegen. 
Eine gute Schulung der Mitarbeiter und ein sicherer Ort zum Schutz der Daten sind immer  wichtiger. 
Die Schwachstellen der höchsten Risiken sind ein unsicheres Passwort-Account, Zweckentfremdung der Zugangsinformationen der Mitarbeiter, wichtige Daten und Akten in ungesicherten Orten speichern. Durch Abschaltung dieser Schwachstellen werden die Forschungsdaten, Produktionsdaten und Akten schwer gestohlen werden.

\begin{table}[H]
	\sffamily
	\caption{Server-Netzlabor-PC-Pool vs. Bedrohungen}
	\tabulinesep = 1mm %bringt die Reihen etwas weiter auseinander, angenehmer zu lesen
	\centering
		\begin{tabu} to 0.9\textwidth {| X[1] | X[1] | X[1] | X[1] | X[1] | X[1] | X[1] |}
		\hline
		\textbf{Wert} & \textbf{Bedrohung} & \textbf{Schwachstelle} & \textbf{Wahrscheinlichkeit} & \textbf{Schaden} & \textbf{Risiko} & \textbf{Maßnahme}\\
		\hline 
		Kucera, Adam &  & & & & &\\
		\hline
	\end{tabu}
\end{table}

Naturkatastrophen: Hochwasser, Hurrikans und Feuerbrände sind auch Bedrohungen für die Server, Netzlabore, PC-Pools und die Infrastruktur. Obwohl die Wahrscheinlichkeit das eine Katastrophe eintritt, nicht enorm hoch ist. 
Die Ortsanalyse für die Gründung und Niederlassung eines Unternehmens, ist ein entscheidender Faktor. In den Beispielen dieses Projektes sind die Orte Görlitz (Polen als Nachbarland) und der östliche Teil Chinas (Experten des Kopierens) aufgeführt.
