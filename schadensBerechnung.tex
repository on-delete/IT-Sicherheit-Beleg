\section{Schadensberechnung}
Anhand der MisUse Cases und Attack-Trees aus den vorigen Kapiteln werden die Risiko-Analyse-Tabellen  aufgestellt. Dabei werden die folgenden Rahmenbedingungen in die Betrachtung eingezogen:
\begin{itemize}
\item ...keine 24 Stunden Videoüberwachung haben.
\item ...kein Sicherheitsdienst vor der Eingangstür haben.
\item ...Chipkartenschlösser vor den wichtigen Räumen haben.
\item ...Unterschiedliche manuelle Schlüssel an den Büros haben.
\item ...Personal und Mitarbeitern die Sicherheitsrechtlinien beigebracht haben.
\end{itemize}

\noindent
Risiko-Analyse Tabellen sind nach folgenden Punkten aufgebaut:
\begin{itemize}
\item Identifikation der Werte.
\item Identifikation der Bedrohungen des Wertes.
\item Identifikation der Schwachstelle der Bedrohung.
\item Mit welcher Wahrscheinlichkeit tritt eine Bedrohung ein? 
\item Welche Auswirkung hat die Bedrohung?
\item Abschätzung der Konsequenzen.
\item Abschätzung  der Kosten des materiellen Schadens.
\item Identifikation der Maßnahmen.
\item Die Skala der Auswertung liegt zwischen 1 (niedrigste Kennziffer) und 5 (höchste Kennziffer). 
\item Die Skala des Risikos liegt zwischen 1 (niedrigste Kennziffer) und 25 (höchste Kennziffer), es ist die Vermehrung der Wahrscheinlichkeit mal der Schaden.
\end{itemize}

\subsection{Ergebnisse}
Im Folgenden sind die Ergebnisse für AutoVit und Fakultät Informatik an der HSZG dargestellt. Die Daten der Risiko Analyse beider Einrichtungen sind in etwa identisch, weil beide ähnliche Werte und Bedrohungen haben.

\begin{table}[H]
\caption{Für die Bedrohung Einbruch}
\begin{tabu} to \linewidth {X[-1,l,m] X[-1,l,m] X[-1,l,m] X[-1,l,m] X[-1,l,m] }
\toprule 
\textbf{Schwachstelle} & \textbf{Wahrschein-\linebreak{}lichkeit} & \textbf{Schaden} & \textbf{Risiko} & \textbf{Maßnahme}\tabularnewline
\midrule 
freier Zugang auf Firmengelände & 4 & 2 & 8 & Zaun bauen, Sichere Zugang\tabularnewline
keine 24 Stunden Videoüberwachung & 4 & 2 & 8 & Videoüberwachung\tabularnewline
keine/schlechte Überprüfung des Inventars & 2 & 4 & 8 & Regelmäßige Kontrolle des Inventars\tabularnewline
Generalschlüssel für alle Mitarbeiter & 2 & 3 & 6 & Sicherheitsfenster\tabularnewline
ungesicherte Türen & 3 & 3 & 9 & Sicherheitstüren \& Schulung\tabularnewline
Ausgeschaltete Sicherheitsmaßnahmen & 3 & 4 & 12 & Überwachung der Mitarbeiter\tabularnewline
\bottomrule 
\end{tabu}
\end{table}

Die Wahrscheinlichkeit, dass ein Einbruch durch freien Zugang auf das Firmengelände und nicht vorhanden sein von 24 Stunden Videoüberwachung passiert, ist hoch. Der Schaden, der angerichtet werden kann, ist verhältnismäßig gering. Daraus resultiert ein niedriges Gesamtrisiko.
Die Wahrscheinlichkeit, dass es keine oder eine schlechte Überprüfung des Inventars gibt, ist niedrig, weil der Hausmeister es regelmäßig kontrolliert, aber die Schäden sind hoch, weil das Material sehr viel wert ist. Daraus resultiert ein niedriges Gesamtrisiko.
Die Wahrscheinlichkeit, dass ein Einbruch durch Generalschlüssel für alle Mitarbeiter und  ungesicherte Türen passiert, ist niedrig und mittelmäßig, weil die meisten Büros ihr eigenes Schloss haben, die Türen haben keine besondere Sicherheitsvorrichtung. Der Schaden, der angerichtet werden kann ist verhältnismäßig gering. Daraus resultiert ein niedriges Gesamtrisiko.
Die Maßnahme um die oben genannten Bedrohungen zu vermeiden sind, einen sicheren Eingang oder Zaun bauen, Videoüberwachung einstellen, das Inventar kontrollieren, Sicherheitsvorrichtungen an den Türen montieren und Schulung für die Mitarbeiter durchzuführen.

\begin{table}[H]
\caption{Für die Bedrohung Diebstahl der Hardware}
\begin{tabu} to \linewidth {X[-1,l,m] X[-1,l,m] X[-1,l,m] X[-1,l,m] X[-1,l,m] }
\toprule 
\textbf{Schwachstelle} & \textbf{Wahrschein-\linebreak{}lichkeit} & \textbf{Schaden} & \textbf{Risiko} & \textbf{Maßnahme}\tabularnewline
\midrule 
Unsichere Firmenhardware & 3 & 5 & 15 & Sicherheits-\linebreak{}richtlinien von der Leitung\tabularnewline
Firmendaten auf BYOD Gerät & 2 & 5 & 10 & Sicherheits-\linebreak{}richtlinien von der Leitung\tabularnewline
\bottomrule 
\end{tabu}
\end{table}

\begin{table}[H]
\caption{Für die Bedrohung Malware}
\begin{tabu} to \linewidth {X[-1,l,m] X[-1,l,m] X[-1,l,m] X[-1,l,m] X[-1,l,m] }
\toprule 
\textbf{Schwachstelle} & \textbf{Wahrschein-\linebreak{}lichkeit} & \textbf{Schaden} & \textbf{Risiko} & \textbf{Maßnahme}\tabularnewline
\midrule 
Unsichere Firmenhardware & 2 & 4 & 8 & Sicherheits-\linebreak{}richtlinien von der Leitung\tabularnewline
Firmendaten auf BYOD Geräten & 3 & 5 & 15 & Sicherheits-\linebreak{}richtlinien von der Leitung\tabularnewline
\bottomrule 
\end{tabu}
\end{table}

Mit einer unsicheren Firmenhardware und Firmendaten auf BYOD\footnote{Bring Your Own Device: Bezeichnung für private mobile Endgeräte wie Laptops oder Smartphones} Geräten ist das Risiko am höchsten, obwohl die Wahrscheinlichkeiten, dass die oben genannten Schwachstellen eintreten, niedrig sind. Die Leitung der Firma/Hochschule hat die Mitarbeiter bezüglich der Sicherheitsrichtlinien zu schulen.

\begin{table}[H]
\caption{Für die Bedrohung Serverkonfiguration}
\begin{tabu} to \linewidth {X[-1,l,m] X[-1,l,m] X[-1,l,m] X[-1,l,m] X[-1,l,m] }
\toprule 
\textbf{Schwachstelle} & \textbf{Wahrschein-\linebreak{}lichkeit} & \textbf{Schaden} & \textbf{Risiko} & \textbf{Maßnahme}\tabularnewline
\midrule 
Default Accounts mit Default Passwörtern scannen & 1 & 4 & 4 & Default Accounts deaktivieren\tabularnewline
unnötige/unsichere Dienste & 1 & 4 & 4 & unnötige Dienste deaktivieren\tabularnewline
fehlende/schlecht konfigurierte Firewall & 2 & 5 & 10 & Intrusion Detektion System\tabularnewline
Exploits durch fehlende Updates & 2 & 3 & 6 & System regelmäßig updaten/Virenscanner\tabularnewline
DDos Angriffe & 2 & 5 & 10 & Schutzfunktionen des Systems einrichten\tabularnewline
\bottomrule
\end{tabu}
\end{table}

Im Prinzip muss ein Server richtig konfiguriert werden, andernfalls ist die Wahrscheinlichkeit, dass die genannten Bedrohungen eintreten noch größer. Die höchsten Gesamtrisiken der Serverkonfiguration sind Angriffe eines unsicheren Dienstes und Malware-Einschleusungen durch fehlende oder schlechte Firewall-Konfigurationen. Gezielte Angriffe auf das System durch die schlechte Schulung der Mitarbeiter, stellen auch ein sehr hohes Risiko dar. DDos Attacken können durch fehlende Schutzmechanismen ausgelöst werden.

\begin{table}[H]
\caption{Für die Bedrohung Menschliches Versagen}
\begin{tabu} to \linewidth {X[-1,l,m] X[-1,l,m] X[-1,l,m] X[-1,l,m] X[-1,l,m] }
\toprule 
\textbf{Schwachstelle} & \textbf{Wahrschein-\linebreak{}lichkeit} & \textbf{Schaden} & \textbf{Risiko} & \textbf{Maßnahme}\tabularnewline
\midrule 
Ausschalten von Sicherheitsmaßnahmen (aus böser Absicht oder Bequemlichkeit z.B. Türen offen lassen) & 3 & 3 & 9 & Überwachung/Schulung/Sicherheitsrichtlinien\tabularnewline
Unsichere Passwörter & 4 & 4 & 16 & Schulung/\linebreak{}Sicherheits-\linebreak{}richtlinien\tabularnewline
Social Engineering & 3 & 4 & 12 & gezielte Schulungen\tabularnewline
\bottomrule
\end{tabu}
\end{table}

Eine professionelle Schulung der Mitarbeiter kann Risiken vermeiden. Unsichere Passwörter von Mitarbeitern sind anfälliger für Angriffe von Außen. Durch Social Engineering können Dritte Personen an firmeninterne Daten gelangen, die zweckentfremdet werden können.

\begin{table}[H]
\caption{Für die Bedrohung Industriespionage}
\begin{tabu} to \linewidth {X[-1,l,m] X[-1,l,m] X[-1,l,m] X[-1,l,m] X[-1,l,m] }
\toprule 
\textbf{Schwachstelle} & \textbf{Wahrschein-\linebreak{}lichkeit} & \textbf{Schaden} & \textbf{Risiko} & \textbf{Maßnahme}\tabularnewline
\midrule 
Mitarbeiter verkaufen interne Daten aufgrund schlechter Bezahlung & 2 & 4 & 8 & Kündigung
Strafanzeige evtl. Haftstrafe
\tabularnewline
Daten werden gestohlen aufgrund unsicherer Passwörter & 3 & 5 & 15 & Schulung/\linebreak{}Sicherheits-\linebreak{}richtlinien\tabularnewline
Physischer Diebstahl durch ungesicherte Büros & 3 & 5 & 15 & sicherer Ort für Unterlagen\tabularnewline
\bottomrule
\end{tabu}
\end{table}

Heutzutage sind sowohl die Industriespionage als auch Diebstähle deutlich gestiegen. Eine gute Schulung der Mitarbeiter und ein sicherer Ort zum Schutz der Daten sind immer wichtiger. Die Schwachstellen der höchsten Risiken sind unsichere Passwörter, Zweckentfremdung der Zugangsinformationen der Mitarbeiter, sowie die unsichere Lagerung wichtiger Daten und Akten.

\begin{table}[H]
\caption{Für die Bedrohung höhere Gewalt}
\begin{tabu} to \linewidth {X[-1,l,m] X[-1,l,m] X[-1,l,m] X[-1,l,m] X[-1,l,m] }
\toprule 
\textbf{Schwachstelle} & \textbf{Wahrschein-\linebreak{}lichkeit} & \textbf{Schaden} & \textbf{Risiko} & \textbf{Maßnahme}\tabularnewline
\midrule 
Feuer durch schlechten Brandschutz & 1 & 5 & 5 & Brandschutz-\linebreak{}anlagen\tabularnewline
Wasserschäden durch z.B. Hochwasser & 1 & 2 & 2 & entsprechende Schutzmaßnahmen\tabularnewline
Diebstahl & 2 & 4 & 8 & Überwachung/Schutzrichtlinien\tabularnewline
\bottomrule
\end{tabu}
\end{table}

Naturkatastrophen: Hochwasser, Hurrikans und Feuerbrände sind auch Bedrohungen für die Server, Netzlabore, PC-Pools und die Infrastruktur. Obwohl die Wahrscheinlichkeit das eine Katastrophe eintritt, nicht enorm hoch ist. Die Ortsanalyse für die Gründung und Niederlassung eines Unternehmens, ist ein entscheidender Faktor. In den Beispielen dieses Projektes sind die Orte Görlitz (Polen als Nachbarland) und der östliche Teil Chinas (Industriespionage) aufgeführt.
